
\section{Introduction}

Sensitivity analysis was performed with respect to various key processes and 
parameters affecting long-term post-closure performance of geologic repositories 
in clay media. Based on the detailed computational Clay 
\gls{GDSM} developed by the \gls{UFD} campaign \cite{clayton_generic_2011}, 
these results provide an overview of the relative importance of processes 
that affect the repository performance of a generic clay disposal concept model. 
Further analysis supports a basis for development of rapid abstracted models in 
the context of system level fuel cycle simulation.

This work is not intended to give an assessment of the performance of a disposal 
system. Rather, it is intended to  generically identify properties and 
parameters expected to influence repository performance in a clay disposal 
envrionment.  Processes and parameters investigated  include the rate of waste 
form degradation, timing of waste package failure, and various coupled 
geochemical and hydrologic characteristics of the natural system including 
diffusion, solubility, and advection. 

This analysis found that diffusion, sorption, and dissolution processes dominate 
dose-related repository perfomance . . . etc.
