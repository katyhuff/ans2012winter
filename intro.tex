
\section{Introduction}



Why is this work important? 

As the United States and other nuclear nations consider alternative fuel cycles and waste disposal options simultaneously, integrated fuel cycle and generic disposal system analysis grows increasingly necessary for informing spent nuclear fuel management policy. 
The long term performance characteristics of deep geologic disposal concepts are affected by heat and radionuclide release characteristics sensitive to disposal system choices as well as variable spent fuel compositions associated with alternative fuel cycles.
Sensitivity analyses using a detailed toolset have identified the dominant physics of candidate geologic host media, repository designs, and engineering components. 
In addition to highlighting areas of need for R&D in homogeneous reducing geologies, this work informs repository analysis in the context of fuel cycle options.
By abstraction of more detailed models, this work captures the dominant physics of radionuclide and heat transport phenomena affecting repository performance in various geologic media and as a function of arbitrary spent fuel composition.

The development of sustainable nuclear fuel cycles is a key challenge as the use of nuclear power expands domestically and internationally. Accordingly, the United States and other nations are considering a number of nuclear fuel cycle and geologic disposal options simultaneously [29, 61]. These decisions are technologically coupled since radionuclide containment performance of a geologic repository is, in part, a function of spent fuel and high level waste composition, which varies among fuel cycle options.

What question are you asking? 

Within a coupled, multiphysics simulation of a canonical repository concept what are the key parameters and dominant physics contributing to repository performance?
In the context of competing, coupled parameters, what is the character of the relationship between those parameters and repository performance?
To what extent could fuel cycle decisions, such as separation and transmutation strategies, affect performance?


What is the existing body of work on this topic?


domestically, focused on yucca mountain
swift & nutt
generic geologic media
red-impact international experimental analysis on specific sites 

How does your work fit into the existing body of literature on this topic? 

Parameters of particular interest in fuel cycle systems analysis have historically been those related to the front end of the fuel cycle [citation please]. However, parameters representing decisions concerning the back end of the fuel cycle are of increasing interest as repositories are being considered internationally and as the United States further investigates repository alternatives to the Yucca Mountain Repository Site (YMR). Choices such as geologic media, engineered barriers, appropriate loading strategies and schedules are all independent parameters up for debate. Due to the coupled nature of repository capacity and performance, these parameters are coupled with decisions about the fuel cycle.


Why is this work needed?

For this reason, sensitivity studies of a generic disposal model is necessary to illuminate performance distinctions of candidate repository host media, designs, and engineering components in the context of varying spent fuel compositions. 
