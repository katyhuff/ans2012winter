
\section{Introduction}

The development of sustainable nuclear fuel cycles is a key challenge as the use 
of nuclear power expands domestically and internationally. Accordingly, the 
United States and other nations are considering a number of nuclear fuel cycle 
and geologic disposal options simultaneously \cite{}. These decisions are 
technologically coupled since radionuclide containment performance of a geologic 
repository is, in part, a function of spent fuel and high level waste 
composition, which varies among fuel cycle options. Accordingly, integrated 
fuel cycle and disposal system analysis grows increasingly necessary for 
informing spent nuclear fuel management policy.  This work enables that analysis 
by identifying dominant physics of radionuclide and heat transport phenomena 
affecting repository performance in homogeneous, reducing geologic media and as 
a function of spent fuel composition. 
%The long term performance characteristics of deep geologic disposal concepts 
%are affected by heat and radionuclide release characteristics sensitive to 
%disposal system choices as well as variable spent fuel compositions associated 
%with alternative fuel cycles. 

Sensitivity analyses using a detailed multiphysics simulation have characterised 
key performance parameters in a candidate repository concept. This concept 
includes a homogeneous, reducing geologic host media, and limited lifetime 
engineered barrier components.  This sensitivity analysis characterizes the 
sensitivity of repository performance to various site and design parameters in 
the context of competing and coupled physics. Finally, this work informs the 
extent to which fuel cycle decisions, such as separation and transmutation 
strategies, may affect performance.


Previous work in the United States has focused on the oxidizing, fractured 
granite tuff of Yucca Mountain. International work has focused more on reducing 
and homogeneous geologies, but most sensitivity analysis has been site-specific 
\cite{swift/nutt, redimpact}.

Parameters of particular interest in fuel cycle systems analysis have 
historically been those related to the front end of the fuel cycle [citation 
please]. However, parameters representing decisions concerning the back end of 
the fuel cycle are of increasing interest as repositories are being considered 
internationally and as the United States further investigates repository 
alternatives to the Yucca Mountain Repository Site (YMR). Choices such as 
geologic media, engineered barriers, appropriate loading strategies and 
schedules are all independent parameters up for debate. Due to the coupled 
nature of repository capacity and performance, these parameters are coupled with 
decisions about the fuel cycle.  For this reason, sensitivity studies of a 
generic disposal model is necessary to illuminate performance distinctions of 
candidate repository host media, designs, and engineering components in the 
context of varying spent fuel compositions. 
