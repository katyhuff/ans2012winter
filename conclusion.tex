\section{Conclusion}

Repository site and engineered barrier selection have historically been 
politically fraught topics.  Pinpointing the important physics for this complex 
multiphysics system can guide R\&D as well as decision making.  While this 
repository concept had only very small releases for all physically realistic 
values, some hydrologic and geochemical parameter strongly influenced transport. 
Meanwhile, engineered barriers were less important for attenuation of transport. 
When varied over a conceivable range, key parameters included many of those 
analyzed, but waste form lifetime was unimportant due to the long 
advective-diffusive pathway in the host media. Solubility and sorption had 
strong effects, indicating that repository geochemistry as well as the inventory 
of high-solubility, low-sorption radionuclides could drive waste treatment 
goals. Unfortunately, highly-soluble, low-sorption radionuclides in clay 
typically include $^{129}$I, for which there is very little available strategy 
for transmutation or alternative use. Additionally, advective and diffusive 
characteristics of the host geology have distinctly varied importance for 
different isotopes, so repository hydrology should also inform waste treatment 
goals in the future. 
