
%\begin{keywords}
%Nuclear fuel cycle, Repository, Hydrologic Contaminant Transport, 
%Clay.
%\end{keywords}
\begin{abstract}

% Every paper must have an abstract.  The abstract must be less than one page 
% long, double-spaced, and in 12-point font size. 

% the importance of these results as new information, 
% and the major conclusions.  

% It must summarize the content of the paper 
Sensitivity analyses identifying key processes and parameters in saturated, 
homogeneous, clay repository environments were performed. 
% and point out the main objectives, 
The parametric relationships described in this work inform the importance of 
repository site and engineered barrier features in the context of fuel cycle 
decisions. This work characterized per-isotope contaminant transport sensitivity 
to geochemical and hydrologic factrs such as solubility, sorption, 
diffusivity, and vertical advective velocity. It also characterized performance 
sensitivity to engineered barrier performance expectations such as waste form 
degradation rate and the time waste package failure. 
% the methods employed, 
Individual and coupled parameter analyses were undertaken within the GoldSim 
simulation software, which tracks the movement of key radionuclides through the 
natural system and engineered barriers. The per-isotope resolution of the 
results inform the potential impacts of fuel cycle decisions on repository 
performance.
% the results obtained
The results here provide an overview of the relative importance of processes 
determining repository performance of a simplified generic disposal concept in 
saturated, homogeneous environments.  Thresholds between primarily diffusive and 
primarily advective transport were identified and the relative influences of 
geological and engineered barrier parameters on the mobility of dose 
contributing isotopes were characterized.  
%  General and well-known information should not be included in the abstract.
\end{abstract}
