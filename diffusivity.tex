
\subsection{Case I : Diffusion Coefficient and Inventory }
\label{sec:diffusivity}

In clay media, diffusion dominates far-field hydrologic transport due to 
characteristically low hydraulic head gradients and permeability. 

The sensitivity of the peak dose to the effective diffusion coefficient, 
$D_{eff}$, in the host rock was analyzed in conjunction with the inventory 
total mass. The parametric ranges for these variables (see Table 
\ref{tab:Cases}), were chosen to cover the full range of expected diffusion 
coefficients in clay and inventories in current wasteforms.

In order to isolate the effect of the far-field behavior, the waste form 
degradation rate was set to be very high as were the solubility and advective 
flow rate through the  \gls{EBS}. This unhindered contaminant flowthrough in the 
near-field and therefore far-field transport was the sole remaining physical 
barrier to release.

Peak dose due to highly soluble, non-sorbing elements such as $I$ and $Cl$ 
was found to be proportional to the radionuclide inventory and 
to the relative diffusivity. 

For sorbing and solubility-limited elements,
two diffusion coefficient regimes were visible in the results of this analysis. 
In the low diffusion coefficient regime, the diffusive pathway through the 
homogeneous permeable porous medium in the far-field continues to be a  dominant 
barrier to nuclide release for normal (non-intrusive) repository conditions.  
In the second regime, for very high diffusion coefficients, the effects of 
additional attenuation phenomena in the natural system can be seen. 

The dependence of peak annual dose on mass factor, the inventory multiplier that 
was varied, was consistently directly 
proportional for all isotopic groups, though slight attenuation of this 
dependence was seen for higher mass factors.
