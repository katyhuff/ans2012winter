
\subsection{Diffusion Coefficient of Far Field}
\label{sec:diffusivity}

In clay media, diffusion dominates far field hydrogeologic transport due to 
characteristically low hydraulic head gradients and permeability. Thus, the 
effective diffusion coefficient is a parameter to which repository performance 
in clay media is very sensitive. 

The sensitivity of the peak dose to the diffusion coefficient in the 
host rock was analyzed. 
The waste inventory total mass was also altered for each value of the diffusion  
coefficient that was sampled. That is, the radionuclide inventory in a reference 
\gls{MTHM} of commercial spent nuclear fuel was multiplied by a scalar mass factor 
to cover the full range of inventories in current wasteforms.

In order to isolate the effect of the far field behavior, the waste form 
degradation rate was set to be very high as were the solubility and advective 
flow rate through the  \gls{EBS}. This unhindered contaminant flowthrough 
in the near field and left far field transport as the sole remaining physical 
barrier to release.

The peak doses due to highly soluble, non-sorbing elements such as $I$ and $Cl$, 
were found to be proportional to the radionuclide inventory and 
largely directly proportional to the relative diffusivity. In the absence of 
solubility limitation and sorption, the peak dose was shown to be directly 
proportional to mass factor. 

With the exception of those dose-contributors assumed to be completely soluble, 
two regimes were visible in the results of this analysis. In the low diffusion 
coefficient regime, the diffusive pathway through the homogeneous permeable 
porous medium in the far field continues to be a  dominant barrier to nuclide 
release for normal (non-intrusive) repository conditions. 

In the second regime, for very high diffusion coefficients, the effects of 
additional attenuation phenomena in the natural system can be seen.  The 
dependence of peak annual dose on mass factor was consistently directly 
proportional for all isotopic groups.

The peak doses due to solubility limited, sorbing elements such as $Np$ and 
$Tc$ demonstrate two major regimes. In the first regime, for 
low values of mass factor, the mean of the peak annual dose rates is directly 
proportional to both reference diffusivity and mass factor.  For higher values 
of mass factor, the sensitivity to reference diffusivity and mass factor are 
both attenuated at higher values.  The attenuation in these regimes 
is due to natural system attenuation, most notably, sorption.

