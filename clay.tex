
\section{Description of the Actual Work}

% used existing gdsms 
These analyses were performed using the Clay \gls{GDSM} developed by the 
\gls{UFD} campaign\cite{clayton_generic_2011}. The Clay \gls{GDSM} is built on the 
GoldSim simulation framework and contaminant transport model. 

This radionuclide transport toolset simulates chemical and physical attenuation 
processes including radionuclide solubility, dispersion phenomena, and 
reversible sorption \cite{golder_goldsim_2010, golder_goldsim_ct_2010}.  
Probabilistic elements of the GoldSim modelling framework enable the models to 
incorporate simple probabilistic \gls{FEPs} that affect repository performance 
including waste package failure, waste form dissolution, and an optional 
vertical advective fast pathway \cite{clayton_generic_2011}.

The disposal concept modeled by the Clay \gls{GDSM} includes an \gls{EBS} which 
can undergo rate based dissolution and barrier failure. Releases from the \gls{EBS} enter 
near field and subsequently far field host rock regions in which diffusive and 
advective transport take place, attenuated by solubility limits as well as 
sorption and dispersion phenomena \cite{clayton_generic_2011}.

\subsection{Mean of the Peak Annual Dose}

In this analysis, repository performance is quantified by radiation dose to a 
hypothetical receptor. Specifically, this sensitivity analysis focuses 
on parameters that affect the mean of the peak annual dose.  The mean of the 
peak annual dose,

\begin{align} \label{MoP}
  D_{MoP,i} &= \frac{\sum_{r=1}^{N}{\max\left[\left.D_{r,i}(t)\right|_{\forall t}\right]}}{N}
  \intertext{where}
  D_{MoP,i} &= \mbox{mean of the peak annual dose due to isotope i} [mrem/yr]\nonumber\\
  D_{i}(t) &= \mbox{annual dose in realization r at time t due to isotope i } [mrem/yr]\nonumber\\
  N &= \mbox{Number of realizations, } \nonumber
\end{align}

is a conservative metric of repository performance and should not be confused 
with the peak of the mean annual dose,

\begin{align} \label{PoM}
  D_{PoM,i} &= \max\left[{\frac{\sum_{r=1}^{N}{\left.D_{r,i}(t)\right|_{\forall t}}}{N}}\right]\\
            &= \mbox{peak of the mean annual dose due to isotope i } [mrem/yr].\nonumber
\end{align}

\subsection{Sampling Scheme}

This analysis has undertaken an analysis strategy to develop a many dimensional 
overview of the key factors in modeled repository performance. To acheive this, 
both individual and dual parametric studies were performed.

Individual parameter studies varied a single parameter of interest in 
detail over a broad range of values. Dual parameter sensitivity studies were 
performed for pairs of parameters expected to exhibit some covariance. For 
each parameter or pair of parameters, forty simulation 
groups varied the parameter or parameters within the range considered. 

For each simulation group, a 100 realization simulation was completed. Each
realization held the parameters being analyzed as constant and sampled 
stochastic values for uncertain parameters not being studied.  A sampling scheme 
developed in previous generic disposal media modeling was implemented in this 
model in order to ensure that the each 100 realization simulation sampled 
identical values for uncertain parameters \cite{clayton_generic_2011, 
nutt_generic_2009}.  

