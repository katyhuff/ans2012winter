
\section{Background: Clay GDSM}

% used existing gdsms 
This analysis utilized the clay \gls{GDSM} developed by the \gls{UFD} campaign 
which performed detailed calculations of radionuclide transport within the GoldSim simulation platform.  GoldSim is a commercial simulation environment \cite{golder_goldsim_2010, golder_goldsim_ct_2010}.  Probabilistic elements of the GoldSim modelling framework enable the models to incorporate simple probabilistic \gls{FEPs} that affect repository performance including waste package failure, waste form dissolution, and an optional vertical advective fast pathway \cite{clayton_generic_2011}. 

The GoldSim framework and its contaminant transport module provide a radionuclide transport toolset that the \glspl{GDSM} has utilized to simulate chemical and physical attenuation processes including radionuclide solubility, dispersion phenomena, and reversible sorption \cite{golder_goldsim_2010, golder_goldsim_ct_2010}. 

\subsection{Mean of the Peak Annual Dose}

In this analysis, repository performance is quantified by radiation dose to a 
hypothetical receptor. Specifically, this sensitivity analysis focuses 
on parameters that affect the mean of the peak annual dose.  The mean of the 
peak annual dose,

\begin{align} \label{MoP}
  D_{MoP,i} &= \frac{\sum_{r=1}^{N}{\max\left[\left.D_{r,i}(t)\right|_{\forall t}\right]}}{N}
  \intertext{where}
  D_{MoP,i} &= \mbox{mean of the peak annual dose due to isotope i} [mrem/yr]\nonumber\\
  D_{i}(t) &= \mbox{annual dose in realization r at time t due to isotope i } [mrem/yr]\nonumber\\
  N &= \mbox{Number of realizations, } \nonumber
\end{align}

is a conservative metric of repository performance. The mean of the 
peak annual dose should not be confused with the peak of the mean annual dose,

\begin{align} \label{PoM}
  D_{PoM,i} &= \max\left[{\frac{\sum_{r=1}^{N}{\left.D_{r,i}(t)\right|_{\forall t}}}{N}}\right]\\
            &= \mbox{peak of the mean annual dose due to isotope i } [mrem/yr].\nonumber
\end{align}

The mean of the peak annual dose rate given in equation \eqref{MoP} 
captures trends as well as the peak of the mean annual dose rate given 
in equation \eqref{PoM}. However, the mean of the peaks metric, $D_{MoP,i}$, was 
chosen in this analysis because it is more conservative since it is able to 
capture temporally local dose maxima and consistently reports higher dose values 
than the peak of the means, $D_{PoM,i} $.

\subsection{Sampling Scheme}

The multiple barrier system modeled in the clay \gls{GDSM} calls for a 
multi-faceted sensitivity analysis. The importance of any single component or 
environmental parameter must be analyzed in the context of the full system of 
barrier components and environmental parameters. Thus, this analysis has 
undertaken an analysis strategy to develop a many dimensional overview of the 
key factors in modeled repository performance. 

To address this, both individual and dual parametric studies were performed. 
Individual parameter studies varied a single parameter of interest in 
detail over a broad range of values. Dual parameter sensitivity studies were 
performed for pairs of parameters expected to exhibit some covariance. For 
each parameter or pair of parameters, forty simulation 
groups varied the parameter or parameters within the range considered. 

For each simulation group, a 100 realization simulation was completed. Each
realization held the parameters being analyzed as constant and sampled 
stochastic values for uncertain parameters not being studied.  A sampling scheme 
developed in previous generic disposal media modeling was implemented in this 
model in order to ensure that the each 100 realization simulation sampled 
identical values for uncertain parameters \cite{clayton_generic_2011, 
nutt_generic_2009}.  

In order to independently analyze the dose contributions from radioisotope 
groups, actinide chains were run independently. This allowed an evaluation of 
the importance of daughter production from distinct actinide chains.

These analyses were performed using the Clay \gls{GDSM} developed by the 
\gls{UFD} campaign\cite{clayton_generic_2011}. The Clay \gls{GDSM} is built on the 
GoldSim software and tracks the movement of key radionuclides through the 
natural system and engineered barriers \cite{golder_goldsim_2010, 
golder_goldsim_ct_2010}.

The disposal concept modeled by the Clay \gls{GDSM} includes an \gls{EBS} which 
can undergo rate based dissolution and barrier failure. Releases from the \gls{EBS} enter 
near field and subsequently far field host rock regions in which diffusive and 
advective transport take place, attenuated by solubility limits as well as 
sorption and dispersion phenomena.  

The Clay \gls{GDSM} models a single waste form, a waste package, additional 
\glspl{EBS}, 
an \gls{EDZ}, and a far field zone using a batch reactor mixing cell framework. This waste unit cell is modeled 
with boundary conditions such that it may be repeated assuming an infinite 
repository configuration. The waste form and engineered barrier system are modeled as well-mixed volumes 
and radial transport away from the cylindrical base case unit cell is modeled as  
one dimensional. Two radionuclide release pathways are considered. One is the nominal, 
undisturbed case, while the other is a fast pathway capable of simulating a 
hypothetical disturbed case 
\cite{clayton_generic_2011}.
