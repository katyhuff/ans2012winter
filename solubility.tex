
\subsection{Solubility Coefficients}

This study varied the solubility coefficients for each isotope in the simulation 

The dissolution behavior of a solute in an aqueous solutions is called its 
solubility. This behavior is limited by the solute's solubility limit, described  
by an equilibrium constant that depends upon temperature, water chemistry, and 
the properties of the element. The equilibrium constant for many reactions are known, and can be found in 
chemical tables.

When $IAP/K<1$, the solution is undersaturated with respect to the products. When, 
conversely, $IAP/K>1$, the solution is oversaturated and precipitation of solids 
in the volume will occur. 

The solubility coefficients were varied in this simulation using a multiplier. 
This multiplier multiplied the most likely values of solubility for each 
element, so the relative solubility between elements was preserved.  Forty 
values of solubility coefficient multiplier were used to change the far field 
solubility.  The values of the solubility multiplier were deliberately varied 
over many magnitudes, from $1\time10^{-9}$ through $5\times10^{-10}$. 

For solubility limits below a certain threshold, the dose releases were directly 
proportional to the solubility limit, indicating that the radionuclide 
concentration saturated the groundwater up to the solubility limit near the 
waste form.  For solubility limits above the threshold, however, further 
increase to the limit had no effect on the peak dose. This demonstrates the 
situation in which the solubility limit is so high that even complete 
dissolution of the waste inventory into the pore water is insufficient to reach 
the solubility limit.

In Figures \ref{fig:SolSumFactor} and \ref{fig:SolSum}, it is clear that for 
solubility constants lower than a threshold, the relationship between peak 
annual dose and solubility limit is strong.

