
\subsection{Vertical Path Length}

Sensitivity of repository performance to characteristics of an advective 
vertical release pathway are examined here.

The model layout assumes that no vertical advective pathway intersects the waste 
packages. Rather, an optional vertical advective pathway with variable length 
can be modeled near the waste packages. This model feature addresses the concern 
that sufficient host medium damage in the evacuation disturbed zone might 
provide a preferred horizontal pathway out of the confines of the repository 
that intersects a fast vertical pathway in which water flows advectively 
upward.

Comparing the effect of the length of the vertical advective path with the 
diffusion coefficent in the \gls{EDZ} and the far field 
provides a notion of the importance of this release pathway. This analysis 
explores the effect of increasing the damage created in evacuation,  
contributes to providing a higher source term at the base of a vertical 
advective pathway. In so doing, this analysis also provides some insight into 
the threshold between primarily diffusive and primarily advective contaminant 
movement. 



\subsection{Parametric Range}

For each value of diffusion coefficient varied in the \gls{EDZ} and far field, 
the vertical path length was varied from 10 to 500 meters.

\begin{table}[ht!]
\centering
\includegraphics[width=0.7\textwidth]{PathLengthAndDiffCoeff/PathLengthAndDiffCoeffGroups.eps}
\caption{Sets of 100 realizations were run for each for vertical advective path 
length and diffusion coefficient coefficient in this dual sensitivity study.}
\label{tab:PathLengthAndDiffCoeffGroups}
\end{table}

\subsection{Results}

This analysis showed that varying advective pathway length within a reasonable 
range had negligible results on repository performance. It also showed that the 
importance of the length of the fast pathway was unaffected by reference 
diffusivities in the \gls{EDZ}. That is, upon changing the reference diffusivity 
in those media simultaneously with the vertical advective pathway length, no 
effect was seen that could be attributed to variability in the advective path 
length. The only variability in the mean of the peak annual doses was due to 
changes in the diffusivity. For this reason it can be concluded that even in the 
case of significant damage to the \gls{EDZ}, the dominant pathway in this 
scenario is the purely diffusive pathway rather than the vertical advective fast 
pathway.
