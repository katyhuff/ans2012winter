%        File: ansWin2012.tex
%     Created: Friday, June 8, 2012 

\documentclass{anstrans}

%% To use the glossaries acronym package, you'll need to define any acronyms you intend to 
%% use. You can define acronyms with \newacronym{label}[acronym]{written out form}
%% To refer to them in the text use \gls{label}
\usepackage[acronym,toc]{glossaries}
\include{acros}
\makeglossaries


%%%%%%%%%%%%%%%%%%%%%%%%%%%%%%%%%%%
\title{ Key Processes and Parameters in a Generic Clay Disposal System Model}

\author{Kathryn D.~Huff$^{1,2}$, W. Mark~Nutt$^2$}
%% uncomment these next five only if using anstrans
\institute{$^1$Nuclear Engineering \& Engineering Physics Dept., University of 
Wisconsin, Madison, WI, 53706\\
$^2$Nuclear Engineering Division, Argonne National Laboratory, Argonne, IL, 
60439}
\email{khuff@cae.wisc.edu \& wnutt@anl.gov}
\usepackage{graphicx}
\usepackage{booktabs} % nice rules for tables
\usepackage{microtype} % if using PDF
\newcommand{\units}[1] {\:\text{#1}}%
\newcommand{\SN}{S$_N$}%{S$_\text{N}$}%{$S_N$}%

\date{}
%%%%%%%%%%%%%%%%%%%%%%%%%%%%%%%%%%%
\begin{document}
%%%%%%%%%%%%%%%%%%%%%%%%%%%%%%%%%%%%%%%%%%%%%%%%%%%%%%%%%%%%%%%%%%%%%%%%%%%%%%%%

\section{Introduction}

The development of sustainable nuclear fuel cycles is a key challenge as the use 
of nuclear power expands domestically and internationally. Accordingly, the 
United States and other nations are considering a number of nuclear fuel cycle 
and geologic disposal options simultaneously \cite{}. These decisions are 
technologically coupled since radionuclide containment performance of a geologic 
repository is, in part, a function of spent fuel and high level waste 
composition, which varies among fuel cycle options. Accordingly, integrated 
fuel cycle and disposal system analysis grows increasingly necessary for 
informing spent nuclear fuel management policy.  This work enables that analysis 
by identifying dominant physics of radionuclide and heat transport phenomena 
affecting repository performance in homogeneous, reducing geologic media and as 
a function of spent fuel composition. 
%The long term performance characteristics of deep geologic disposal concepts 
%are affected by heat and radionuclide release characteristics sensitive to 
%disposal system choices as well as variable spent fuel compositions associated 
%with alternative fuel cycles. 

Sensitivity analyses using a detailed multiphysics simulation have characterised 
key performance parameters in a candidate repository concept. This concept 
includes a homogeneous, reducing geologic host media, and limited lifetime 
engineered barrier components.  This sensitivity analysis characterizes the 
sensitivity of repository performance to various site and design parameters in 
the context of competing and coupled physics. Finally, this work informs the 
extent to which fuel cycle decisions, such as separation and transmutation 
strategies, may affect performance.


Previous work in the United States has focused on the oxidizing, fractured 
granite tuff of Yucca Mountain. International work has focused more on reducing 
and homogeneous geologies, but most sensitivity analysis has been site-specific 
\cite{swift/nutt, redimpact}.

Parameters of particular interest in fuel cycle systems analysis have 
historically been those related to the front end of the fuel cycle [citation 
please]. However, parameters representing decisions concerning the back end of 
the fuel cycle are of increasing interest as repositories are being considered 
internationally and as the United States further investigates repository 
alternatives to the Yucca Mountain Repository Site (YMR). Choices such as 
geologic media, engineered barriers, appropriate loading strategies and 
schedules are all independent parameters up for debate. Due to the coupled 
nature of repository capacity and performance, these parameters are coupled with 
decisions about the fuel cycle.  For this reason, sensitivity studies of a 
generic disposal model is necessary to illuminate performance distinctions of 
candidate repository host media, designs, and engineering components in the 
context of varying spent fuel compositions. 


%%%%%%%%%%%%%%%%%%%%%%%%%%%%%%%%%%%%%%%%%%%%%%%%%%%%%%%%%%%%%%%%%%%%%%%%%%%%%%%%

\begin{frame}[c]
  \frametitle{Case I : Diffusion Coefficient and Inventory}
The sensitivity of the peak dose to the reference diffusivity of the 
host rock was analyzed. 
That is, the radionuclide inventory in a reference \gls{MTHM} of commercial spent nuclear fuel was multiplied by a scalar mass factor.  
It was expected that changing these two parameters in tandem would capture 
\begin{itemize}
  \item the importance of diffusivity in the far field 
  \item a threshold at which the effect of waste inventory dissolution is attenuated by solubility limits.
\end{itemize}
\end{frame}

\begin{frame}[c]
  \frametitle{Case I : Diffusion Coefficient and Inventory}
\begin{table}[ht!]
\centering
\footnotesize{
\begin{tabular}{|l|l|l|r|r|}
\multicolumn{5}{c}{\textbf{Simulation Cases}}\\
\hline
\textbf{Case} & \textbf{Parameter} & \textbf{Units} & \textbf{Min. Value} & \textbf{Max. Value}\\
\hline
I     & $D_{eff}$    & $[m^2\cdot s^{-1}]$       & $10^{-8}$    &  $10^{-5}$ \\
      & Inventory              & [MTHM]         & $10^{-4}$    &  $10^1$ \\
\hline
\end{tabular}
\caption{Each dual and single parameter simulation case had 40 simulation 
groups of 100 realizations each.}
\label{tab:Cases}
}
\end{table}


\begin{table}[hbp!]
\centering
\includegraphics[width=0.7\textwidth]{DiffCoeffAndInvEBSFail/DiffCoeffAndInvGroups.eps}
\caption{Diffusion coefficient and mass factor simulation groupings.}
\label{tab:DiffCoeffAndInvGroups}
\end{table}
\end{frame}

\begin{frame}[c]
  \frametitle{Case I : Diffusion Coefficient and Inventory}
\begin{figure}[ht]
\centering
\includegraphics[width=\linewidth]{./DiffCoeffAndInvEBSFail/I-129.eps}
\caption{The peak doses due to highly soluble, non-sorbing elements such as $I$ 
are largely directly proportional to the relative diffusivity.  }
\label{fig:DCInvI129}
\end{figure}
\end{frame}

\begin{frame}[c]
  \frametitle{Case I : Diffusion Coefficient and Inventory}
\begin{figure}[ht]
\centering
\includegraphics[width=\linewidth]{DiffCoeffAndInvEBSFail/I-129-MF.eps}
\caption{$^{129}I$ mass factor sensitivity.}
\caption{The peak doses due to highly soluble, non-sorbing elements such as $I$ 
are  proportional to the radionuclide inventory.}
\label{fig:DCInvI129MF}
\end{figure}
\end{frame}

\begin{frame}[c]
  \frametitle{Case I : Diffusion Coefficient and Inventory}

\begin{figure}[ht]
\centering
\includegraphics[width=\linewidth]{DiffCoeffAndInvEBSFail/Cl-36.eps}
\caption{The peak doses due to highly soluble, non-sorbing elements such as $Cl$ 
are largely directly proportional to the relative diffusivity.  }
\label{fig:DCInvCl36}
\end{figure}
\end{frame}

\begin{frame}[c]
  \frametitle{Case I : Diffusion Coefficient and Inventory}

\begin{figure}[ht]
\centering
\includegraphics[width=\linewidth]{DiffCoeffAndInvEBSFail/Cl-36-MF.eps}
\caption{The peak doses due to highly soluble, non-sorbing elements such as $I$ 
are  proportional to the radionuclide inventory.}
\label{fig:DCInvCl36MF}
\end{figure}
\end{frame}


\begin{frame}[c]
  \frametitle{Case I : Diffusion Coefficient and Inventory}
\begin{figure}[ht!]
\centering
\includegraphics[width=\linewidth]{DiffCoeffAndInvEBSFail/Tc-99.eps}
\caption{$^{99}Tc$ relative diffusivity sensitivity.} 
\label{fig:DCInvTc99}
\end{figure}
\end{frame}

\begin{frame}[c]
  \frametitle{Case I : Diffusion Coefficient and Inventory}

\begin{figure}[ht!]
\centering
\includegraphics[width=\linewidth]{DiffCoeffAndInvEBSFail/Tc-99-MF.eps}
\caption{$^{99}Tc$ mass factor sensitivity.}
\label{fig:DCInvTc99MF}
\end{figure}
\end{frame}

\begin{frame}[c]
  \frametitle{Case I : Diffusion Coefficient and Inventory}


\begin{figure}[ht!]
\centering
\includegraphics[width=\linewidth]{DiffCoeffAndInvEBSFail/Np-237.eps}
\caption{$^{237}Np$ relative diffusivity sensitivity.} 
\label{fig:DCInvNp237}
\end{figure}
\end{frame}

\begin{frame}[c]
  \frametitle{Case I : Diffusion Coefficient and Inventory}

\begin{figure}[ht!]
\centering
\includegraphics[width=\linewidth]{DiffCoeffAndInvEBSFail/Np-237-MF.eps}
\caption{$^{237}Np$ mass factor sensitivity.}
\label{fig:DCInvNp237MF}
\end{figure}
\end{frame}

\begin{frame}[c]
  \frametitle{Case II : Vertical Advective Velocity and Diffusion Coefficient}
Advection is transport driven by bulk water velocity while diffusion is the 
result of Brownian motion across concentration gradients.  The method by which 
the dominant solute transport mode (diffusive or advective) is determined for a 
particular porous medium is by use of the dimensionless Peclet number, 

\begin{align} 
  Pe &= \frac{nvL}{\alpha nv + D_{eff}},\\
  &= \frac{\mbox{advective rate}}{\mbox{diffusive rate}}\nonumber
  \intertext{where} 
  n &= \mbox{ solute accessible porosity } [\%]\nonumber\\
  v &= \mbox{ advective velocity } [m\cdot s^{-1}] \nonumber\\
  L &= \mbox{ transport distance } [m]\nonumber\\
  \alpha &= \mbox{ dispersivity } [m]\nonumber\\
  D_{eff} &= \mbox{ effective diffusion coefficient } [m^2\cdot s^{-1}].\nonumber
\end{align}
For a high $Pe$ number, advection is the dominant transport mode, while 
diffusive or dispersive transport dominates for a low $Pe$ number
\cite{schwartz_fundamentals_2004}.
\end{frame}

\begin{frame}[c]
  \frametitle{Case II : Vertical Advective Velocity and Diffusion Coefficient}
  \begin{table}[ht!]
\centering
\footnotesize{
\begin{tabular}{|l|l|l|r|r|}
\multicolumn{5}{c}{\textbf{Simulation Cases}}\\
\hline
\textbf{Case} & \textbf{Parameter} & \textbf{Units} & \textbf{Min. Value} & \textbf{Max. Value}\\
\hline
II    & $V_{adv, y}$ & $[m \cdot yr^{-1}]$       & $6.31\times10^{-8}$  &  $6.31\times10^{-4}$ \\
      & $D_{eff}$    & $[m^2\cdot s^{-1}]$       & $10^{-8}$    &  $10^{-5}$ \\
\hline
\end{tabular}
\caption{Case II varied the advective velocity and effective diffusivity to 
  determine the nature of the threshold between the diffusive and advective 
  regimes. This dual parameter simulation case had 40 simulation 
groups of 100 realizations each.}
\label{tab:Cases}
}
\end{table}



The forty runs are a combination of the five values of the vertical advective 
velocity and eight magnitudes of relative diffusivity.
\begin{table}
\centering
\includegraphics[width=0.8\textwidth]{AdvVelAndDiffCoeffEBSFail/AdvVelAndDiffCoeffGroups.eps}
\caption{Vertical advective velocity and diffusion coefficient simulation groupings.}
\label{tab:AdvVelAndDiffCoeffGroups}
\end{table}
\end{frame}

\begin{frame}[c]
  \frametitle{Case II : Vertical Advective Velocity and Diffusion Coefficient}
\begin{figure}[htp!]
\centering
\includegraphics[width=0.8\textwidth]{AdvVelAndDiffCoeffEBSFail/I-129.eps}
\caption{$^{129}I$. For vertical advective velocities 
$6.31\times10^{-6}[m/yr]$ and above, lower reference diffusivities are 
ineffective at attenuating the mean of the peak doses for soluble, non-sorbing 
elements. 
}
\label{fig:VAdvVelI129}
\end{figure}
\end{frame}

\begin{frame}[c]
  \frametitle{Case II : Vertical Advective Velocity and Diffusion Coefficient}

\begin{figure}[ht!]
\centering
\includegraphics[width=0.8\textwidth]{AdvVelAndDiffCoeffEBSFail/I-129-VAdvVel.eps}
\caption{$^{129}I$.
For vertical advective velocities 
$6.31\times10^{-6}[m/yr]$ and above, lower reference diffusivities are 
ineffective at attenuating the mean of the peak doses for soluble, non-sorbing 
elements. 
}
\label{fig:VAdvVelI129VAdvVel}
\end{figure}
\end{frame}

\begin{frame}[c]
  \frametitle{Case II : Vertical Advective Velocity and Diffusion Coefficient}
\begin{figure}[htp!]
\centering
\includegraphics[width=0.8\textwidth]{AdvVelAndDiffCoeffEBSFail/Np-237.eps}
\caption{$^{237}Np$.
shows a very weak influence on peak annual dose 
rate for low reference diffusivities, but a direct proportionality between 
dose and reference diffusivity above a threshold.}
\label{fig:VAdvVelNp237}
\end{figure}
\end{frame}

\begin{frame}[c]
  \frametitle{Case II : Vertical Advective Velocity and Diffusion Coefficient}
\begin{figure}[ht!]
\centering
\includegraphics[width=0.8\textwidth]{AdvVelAndDiffCoeffEBSFail/Np-237-VAdvVel.eps}
\caption{$^{237}Np$.
shows a very weak influence on peak annual dose 
rate for low reference diffusivities, but a direct proportionality between 
dose and reference diffusivity above a threshold.}
\label{fig:VAdvVelNp237VAdvVel}
\end{figure}
\end{frame}

\begin{frame}[c]
  \frametitle{Case II : Vertical Advective Velocity and Diffusion Coefficient}
\begin{figure}[htp!]
\centering
\includegraphics[width=0.8\textwidth]{AdvVelAndDiffCoeffEBSFail/Se-79.eps}
\caption{$^{79}Se$.
$Se$ is non sorbing, but solubility limited.  
For low vertical advective velocity, 
the system is diffusion dominated.}
\label{fig:VAdvVelSe79}
\end{figure}
\end{frame}

\begin{frame}[c]
  \frametitle{Case II : Vertical Advective Velocity and Diffusion Coefficient}
\begin{figure}[ht!]
\centering
\includegraphics[width=0.8\textwidth]{AdvVelAndDiffCoeffEBSFail/Se-79-VAdvVel.eps}
\caption{$^{79}Se$.
$Se$ is non sorbing, but solubility limited.  
For high vertical advective 
velocity, the diffusivity remains important even in the advective regime as 
spreading facilitates transport in the presence of solubility limited transport. 
vertical advective velocity sensitivity.}
\label{fig:VAdvVelSe79VAdvVel}
\end{figure}
\end{frame}



\begin{frame}[c]
  \frametitle{Case III : Solubility Coefficient}
The reference solubilities for each element were multiplied by the multiplier 
for each simulation group. This technique preserved relative solubility among 
  elements. 

\begin{table}[ht!]
\centering
\footnotesize{
\begin{tabular}{|l|l|l|r|r|}
\multicolumn{5}{c}{\textbf{Simulation Cases}}\\
\hline
\textbf{Case} & \textbf{Parameter} & \textbf{Units} & \textbf{Min. Value} & \textbf{Max. Value}\\
\hline
III   & $S_i$        & $[mol\cdot m^{-3}]$       & $(1\times10^{-9})\langle S_i\rangle $    &  $(5\times10^{10})\langle S_i\rangle $ \\
\hline
\end{tabular}
\caption{Case III varied a solubility factor across many magnitudes. This single parameter simulation case had 40 simulation 
groups of 100 realizations each.}
\label{tab:Cases}
}
\end{table}


\end{frame}

\begin{frame}[c]
  \frametitle{Case III : Solubility Coefficient}


\begin{figure}[ht]
\centering
\includegraphics[width=0.8\textwidth]{Solubility/Solubility_Summary_SolFactor.eps}
\caption{
The peak annual dose due to an inventory, $N$, of each isotope.
For solubility constants lower than the inventory threshold, the relationship between peak 
annual dose and solubility limit is strong.}
\label{fig:SolSumFactor}
\end{figure}
\end{frame}



\begin{frame}[c]
  \frametitle{Case IV : Partition Coefficient}

This analysis investigated the peak dose rate contribution from various 
radionuclides to the partition coefficient of those radionuclides. 

The partition or distribution coefficient, $K_d$, relates the amount of contaminant adsorbed into the 
solid phase of the host medium to the amount of contaminant adsorbed into the 
aqueous phase of the host medium. It is a common empirical coefficient used to 
capture the effects of a number of retardation mechanisms. The coefficient 
$K_d$, in units of $[m^3\cdot kg^{-1}]$, is the ratio of the mass of contaminant in the 
solid to the mass of contaminant in the solution.
\end{frame}

\begin{frame}[c]
  \frametitle{Case IV : Partition Coefficient}
The retardation factor, $R_f$, which is the ratio between velocity of water through a 
volume and the velocity of a contaminant through that volume, can be expressed 
in terms of the partition coefficient,

\begin{align}
  R_f &= 1+\frac{\rho_b}{n_e}K_d
  \intertext{where}
  \rho_b &= ~~\mbox{bulk density}[kg\cdot m^{-3}]\nonumber
  \intertext{and}
  n_e &= ~~\mbox{effective porosity of the medium}[\%].\nonumber
\end{align}
\end{frame}

\begin{frame}[c]
  \frametitle{Case IV : Partition Coefficient}

The parameters in this model were all set to the default values except a multiplier 
applied to the partitioning $K_d$ coefficients.
\begin{table}[ht!]
\centering
\footnotesize{
\begin{tabular}{|l|l|l|r|r|}
\multicolumn{5}{c}{\textbf{Simulation Cases}}\\
\hline
\textbf{Case} & \textbf{Parameter} & \textbf{Units} & \textbf{Min. Value} & \textbf{Max. Value}\\
\hline
IV    & $K_{d,i}$    & $[m^3\cdot kg^{-1}]$       & $(1\times10^{-9})\langle K_{d,i}\rangle $    &  $(5\times10^{10})\langle K_{d,i}\rangle $ \\
\hline
\end{tabular}
\caption{Case IV varied the partitioning coefficient, $K_d$, multiplication 
  factor. This single parameter simulation case had 40 simulation 
groups of 100 realizations each.}
\label{tab:Cases}
}
\end{table}


\end{frame}

\begin{frame}[c]
  \frametitle{Case IV : Partition Coefficient}

\begin{figure}[ht]
\centering
\includegraphics[width=0.8\textwidth]{Sorption/Retardation_Summary_kdFactor.eps}
\caption{
For retardation coefficients greater than a threshold, the 
relationship between peak annual dose and retardation coefficient is a strong 
inverse one. }
\label{fig:KdSumFactor}
\end{figure}
\end{frame}

\begin{frame}[c]
  \frametitle{Case IV : Partition Coefficient}

\begin{figure}[ht]
\centering
\includegraphics[width=0.8\textwidth]{Sorption/Retardation_Summary_kd.eps}
\caption{
For retardation coefficients greater than a threshold, the 
relationship between peak annual dose and retardation coefficient is a strong 
inverse one. }
\label{fig:KdSum}
\end{figure}
\end{frame}



\begin{frame}[c]
  \frametitle{Case V : Waste Form Degradation Rate and Inventory}
These runs varied the waste form degradation rate and the waste inventory mass 
factor.  There were forty runs corresponding to eight values of the waste form degradation 
rate and five values of the mass factor.

\begin{table}[ht!]
\centering
\footnotesize{
\begin{tabular}{|l|l|l|r|r|}
\multicolumn{5}{c}{\textbf{Simulation Cases}}\\
\hline
\textbf{Case} & \textbf{Parameter} & \textbf{Units} & \textbf{Min. Value} & \textbf{Max. Value}\\
\hline
V     & $R_{WFDeg.}$           & $[yr^{-1}]$       & $10^{-9}$    &  $10^{-2}$ \\
      & Inventory              & [MTHM]         & $10^{-4}$    &  $10^1$ \\
\hline
\end{tabular}
\caption{Each dual and single parameter simulation case had 40 simulation 
groups of 100 realizations each.}
\label{tab:Cases}
}
\end{table}


\end{frame}

\begin{frame}[c]
  \frametitle{Case V : Waste Form Degradation Rate and Inventory}

Safety indicators for post closure repository performance have been developed by 
the \gls{UFD} campaign which utilize the inventory multiplier that was varied in 
this study \cite{nutt_generic_2009}. These indicators are normalized by a 
normalization factor (100 mrem/yr) recommended by the \gls{IAEA} as the limit to 
``relevant critical members of the public'' \cite{iaea_international_1996}. The functional form for 
this safety indicator for a single waste category, \gls{HLW}, is just 

\begin{align}
SI_{G} &= \left(\frac{\sum_{i=1}^{N}D_{G,i}(I_i, F_{d})}{100mrem/yr}\right)[GWe/yr].
\label{indicator}
\intertext{where}
SI_{G} &= \mbox{Safety indicator for disposal in media type G}[GWe/yr]\nonumber\\
N &= \mbox{Number of key radionuclides considered in this indicator}\nonumber\\
D_{G,i} &= \mbox{Peak dose rate from isotope i in media type G}[mrem/yr]\nonumber\\
F_{d} &= \mbox{Fractional waste form degradation rate}[1/yr].\nonumber
\end{align}
\end{frame}

\begin{frame}[c]
  \frametitle{Case V : Waste Form Degradation Rate and Inventory}
\begin{figure}[ht!]
\centering
\includegraphics[width=0.8\textwidth]{WFDegAndInv/I-129.eps}
\caption{
Highly soluble and non-sorbing $^{129}I$ demonstrates a direct proportionality between dose rate and 
fractional degradation rate until a turnover where other natural system 
parameters dampen transport.} 
\label{fig:WFDegI129}
\end{figure}
\end{frame}

\begin{frame}[c]
  \frametitle{Case V : Waste Form Degradation Rate and Inventory}

\begin{figure}[ht!]
\centering
\includegraphics[width=0.8\textwidth]{WFDegAndInv/I-129-MF.eps}
\caption{
Highly soluble and non-sorbing $^{129}I$ domonstrates a direct 
proportionality to the inventory multiplier.}
\label{fig:WFDegI129MF}
\end{figure}
\end{frame}

\begin{frame}[c]
  \frametitle{Case V : Waste Form Degradation Rate and Inventory}

\begin{figure}[ht!]
\centering
\includegraphics[width=0.8\textwidth]{WFDegAndInv/Cl-36.eps}
\caption{
Highly soluble and non-sorbing $^{36}Cl$ demonstrates a direct proportionality between dose rate and 
fractional degradation rate until a turnover where other natural system 
parameters dampen transport.} 
\label{fig:WFDegCl36}
\end{figure}

\end{frame}

\begin{frame}[c]
  \frametitle{Case V : Waste Form Degradation Rate and Inventory}
\begin{figure}[ht!]
\centering
\includegraphics[width=0.8\textwidth]{WFDegAndInv/Cl-36-MF.eps}
\caption{
Highly soluble and non-sorbing $^{129}I$ domonstrates a direct 
proportionality to the inventory multiplier.}
\label{fig:WFDegCl36MF}
\end{figure}
\end{frame}

\begin{frame}[c]
  \frametitle{Case V : Waste Form Degradation Rate and Inventory}
The peaks for solubility limited, sorbing elements such as $Tc$ and $Np$, on the 
other hand, have a more dramatic turnover.  For very high degradation rates, the 
dependence on mass factor starts to round off due to attenuation by solubility 
limits.


\begin{figure}[ht!]
\centering
\includegraphics[width=0.8\textwidth]{WFDegAndInv/Tc-99.eps}
\caption{
Solubility limited and sorbing $^{99}Tc$ demonstrates a direct proportionality 
to fractional degradation rate until attuation by its solubility limit and other 
natural system parameters. } 
\label{fig:WFDegTc99}
\end{figure}
\end{frame}

\begin{frame}[c]
  \frametitle{Case V : Waste Form Degradation Rate and Inventory}

\begin{figure}[ht!]
\centering
\includegraphics[width=0.8\textwidth]{WFDegAndInv/Tc-99-MF.eps}
\caption{
  Solubility limited and sorbing $^{99}Tc$ demonstrates a direct proportionality 
to fractional degradation rate until attuation by its solubility limit and other 
natural system parameters. } 
\label{fig:WFDegTc99MF}
\end{figure}

\end{frame}

\begin{frame}[c]
  \frametitle{Case V : Waste Form Degradation Rate and Inventory}

\begin{figure}[ht!]
\centering
\includegraphics[width=0.8\textwidth]{WFDegAndInv/Np-237.eps}
\caption{
  Solubility limited and sorbing $^{237}Np$ demonstrates a direct proportionality 
to fractional degradation rate until attuation by its solubility limit and other 
natural system parameters. } 
\label{fig:WFDegNp237}
\end{figure}
\end{frame}

\begin{frame}[c]
  \frametitle{Case V : Waste Form Degradation Rate and Inventory}

\begin{figure}[ht!]
\centering
\includegraphics[width=0.8\textwidth]{WFDegAndInv/Np-237-MF.eps}
\caption{
  Solubility limited and sorbing $^{237}Np$ demonstrates a direct proportionality 
to fractional degradation rate until attuation by its solubility limit and other 
natural system parameters. } 
\label{fig:WFDegNp237MF}
\end{figure}

\end{frame}


\begin{frame}[c]
  \frametitle{Case VI : Waste Package Failure Time and Diffusion Coefficient}

To investigate the effect of the waste package failure time, it was varied over 
five magnitudes from one thousand to ten million years. Simultaneously, the reference 
diffusivity was varied over the eight magnitudes between $1\times10^{-8}$ and 
$1\times10^{-15}$ in order to determine the correlation between increased 
radionuclide mobility and the waste package lifetime. 
\begin{table}[ht!]
\centering
\footnotesize{
\begin{tabular}{|l|l|l|r|r|}
\multicolumn{5}{c}{\textbf{Simulation Cases}}\\
\hline
\textbf{Case} & \textbf{Parameter} & \textbf{Units} & \textbf{Min. Value} & \textbf{Max. Value}\\
\hline 
VI    & $t_{WPFail}$        & $[yr]$         & $10^3$    &  $10^7$ \\
      & $D_{eff}$           & $[m^2\cdot s^{-1}]$       & $10^{-8}$    &  $10^{-5}$ \\
\hline
\end{tabular}
\caption{Case IV simultaneously varied the time of waste package failure and the 
  reference diffusivity. This dual parameter simulation case had 40 simulation 
groups of 100 realizations each.}
\label{tab:Cases}
}
\end{table}


\end{frame}

\begin{frame}[c]
  \frametitle{Case VI : Waste Package Failure Time and Diffusion Coefficient}

For the clay repository, the waste package failure time is entirely irrelevant 
until waste package failure times reach the million or ten million year time 
scale. 

\begin{figure}[ht!]
\centering
\includegraphics[width=0.8\textwidth]{WPFailExtended/I-129.eps}
\caption{$^{129}I$ waste package failure time sensitivity. }
\label{fig:WPFailI129}
\end{figure}
\end{frame}

\begin{frame}[c]
  \frametitle{Case VI : Waste Package Failure Time and Diffusion Coefficient}

\begin{figure}[ht!]
\centering
\includegraphics[width=0.8\textwidth]{WPFailExtended/I-129-WPFail.eps}
\caption{$^{129}I$ waste package failure time sensitivity. }
\label{fig:WPFailI129}
\end{figure}
\end{frame}





%%%%%%%%%%%%%%%%%%%%%%%%%%%%%%%%%%%%%%%%%%%%%%%%%%%%%%%%%%%%%%%%%%%%%%%%%%%%%%%%
\section{Clay}

\subsection{Diffusion Coefficient of Far Field}
\label{sec:diffusivity}

In clay media, diffusion dominates far field hydrogeologic transport due to characteristically low hydraulic head gradients and permeability. Thus, the effective diffusion coefficient is a parameter to which repository performance in clay media is very sensitive. 

The sensitivity of the peak dose to the reference diffusivity of the 
host rock was analyzed. The diffusion coefficient for all isotopes was varied
The waste inventory total mass was also altered for each value of the diffusion  
coefficient. That is, the radionuclide inventory in a reference 
\gls{MTHM} of commercial spent nuclear fuel was multiplied by a scalar mass factor.  

In order to isolate the effect of the far field behavior, the waste form 
degradation rate was set to be very high as were the solubility and advective 
flow rate through the  \gls{EBS}. This guaranteed that contaminant flowthrough 
in the near field was unhindered, leaving the far field as the dominant barrier 
to release.

The reference diffusivity was varied over the eight magnitudes between $ 
10^{-8}$ and $10^{-15}$ $[m^2 /s]$ .  The Mass Factor, the unitless inventory 
multiplier, was simultaneously varied over the five magnitudes between $10^{-4}$ 
and $10^{1} [-]$. That is, the radionuclide inventory was varied between 
$10^{-4}$ and $10^{1}$ of that in one \gls{MTHM} of \gls{SNF}, which is expected 
to cover the full range of inventories in current wasteforms.

The peak doses due to highly soluble, non-sorbing elements such as $I$ and $Cl$, 
are  proportional to the radionuclide inventory and 
largely directly proportional to the relative diffusivity. 

Long lived $^{129}I$ and $^{36}Cl$ are assumed to have near complete solubility, 
so in Figures \ref{fig:DCInvI129} and \ref{fig:DCInvCl36}, the effect of a 
solubility limited attenuation regime is not seen. Even for very low 
diffusivities, the diffusion length of the far field is the primary barrier. In 
Figures \ref{fig:DCInvI129MF} and \ref{fig:DCInvCl36MF} it is clear that in the 
absence of solubility limitation and sorption, the peak dose is directly 
proportional to mass factor. 

Both $Cl$ and $I$ are soluble and non-sorbing. The amount of $^{129}I$ in the 
\gls{SNF} inventory is greater than the amount of $^{36}Cl$, so a difference in 
magnitudes are expected, however, the trends should be the same. Since the 
halflife of $^{36}Cl$, $3\times10^5[yr]$, is much shorter than the half life of 
$^{129}I$, $1.6\times10^7[yr]$, a stronger proportional dependence on mass 
factor is seen for $Cl$ due to its higher decay rate. 

With the exception of those dose-contributors assumed to be completely soluble, 
two regimes were visible in the results of this analysis. In low diffusion 
coefficient regime, the diffusive pathway through the homogeneous permeable 
porous medium in the far field continues to be a  dominant barrier to nuclide 
release for normal (non-intrusive) repository conditions. 

In the second regime, for very high diffusion coefficients, the effects of 
additional attenuation phenomena in the natural system can be seen.  The 
dependence of peak annual dose on mass factor was consistently directly 
proportional for all isotopic groups.

The peak doses due to solubility limited, sorbing elements such as $Np$ and 
$Tc$ demonstrate two major regimes. In the first regime, for 
low values of mass factor, the mean of the peak annual dose rates is directly 
proportional to both reference diffusivity and mass factor.  For higher values 
of mass factor, the sensitivity to reference diffusivity and mass factor are 
both attenuated at higher values.  The attenuation in these regimes 
is due to natural system attenuation, most notably, sorption.

$^{237}Np$ and $^{99}Tc$ exhibit a strong proportional relationship 
between diffusivity and dose in Figures \ref{fig:DCInvTc99} and 
\ref{fig:DCInvNp237}. This relationship is muted as diffusivity 
increases. Both are directly proportional to mass factor until they reach the 
point of attenuation by their solubility limits, as can be seen in 
Figures \ref{fig:DCInvTc99MF} and \ref{fig:DCInvNp237MF}.



\begin{frame}[c]
  \frametitle{Case II : Vertical Advective Velocity and Diffusion Coefficient}
Advection is transport driven by bulk water velocity while diffusion is the 
result of Brownian motion across concentration gradients.  The method by which 
the dominant solute transport mode (diffusive or advective) is determined for a 
particular porous medium is by use of the dimensionless Peclet number, 

\begin{align} 
  Pe &= \frac{nvL}{\alpha nv + D_{eff}},\\
  &= \frac{\mbox{advective rate}}{\mbox{diffusive rate}}\nonumber
  \intertext{where} 
  n &= \mbox{ solute accessible porosity } [\%]\nonumber\\
  v &= \mbox{ advective velocity } [m\cdot s^{-1}] \nonumber\\
  L &= \mbox{ transport distance } [m]\nonumber\\
  \alpha &= \mbox{ dispersivity } [m]\nonumber\\
  D_{eff} &= \mbox{ effective diffusion coefficient } [m^2\cdot s^{-1}].\nonumber
\end{align}
For a high $Pe$ number, advection is the dominant transport mode, while 
diffusive or dispersive transport dominates for a low $Pe$ number
\cite{schwartz_fundamentals_2004}.
\end{frame}

\begin{frame}[c]
  \frametitle{Case II : Vertical Advective Velocity and Diffusion Coefficient}
  \begin{table}[ht!]
\centering
\footnotesize{
\begin{tabular}{|l|l|l|r|r|}
\multicolumn{5}{c}{\textbf{Simulation Cases}}\\
\hline
\textbf{Case} & \textbf{Parameter} & \textbf{Units} & \textbf{Min. Value} & \textbf{Max. Value}\\
\hline
II    & $V_{adv, y}$ & $[m \cdot yr^{-1}]$       & $6.31\times10^{-8}$  &  $6.31\times10^{-4}$ \\
      & $D_{eff}$    & $[m^2\cdot s^{-1}]$       & $10^{-8}$    &  $10^{-5}$ \\
\hline
\end{tabular}
\caption{Case II varied the advective velocity and effective diffusivity to 
  determine the nature of the threshold between the diffusive and advective 
  regimes. This dual parameter simulation case had 40 simulation 
groups of 100 realizations each.}
\label{tab:Cases}
}
\end{table}



The forty runs are a combination of the five values of the vertical advective 
velocity and eight magnitudes of relative diffusivity.
\begin{table}
\centering
\includegraphics[width=0.8\textwidth]{AdvVelAndDiffCoeffEBSFail/AdvVelAndDiffCoeffGroups.eps}
\caption{Vertical advective velocity and diffusion coefficient simulation groupings.}
\label{tab:AdvVelAndDiffCoeffGroups}
\end{table}
\end{frame}

\begin{frame}[c]
  \frametitle{Case II : Vertical Advective Velocity and Diffusion Coefficient}
\begin{figure}[htp!]
\centering
\includegraphics[width=0.8\textwidth]{AdvVelAndDiffCoeffEBSFail/I-129.eps}
\caption{$^{129}I$. For vertical advective velocities 
$6.31\times10^{-6}[m/yr]$ and above, lower reference diffusivities are 
ineffective at attenuating the mean of the peak doses for soluble, non-sorbing 
elements. 
}
\label{fig:VAdvVelI129}
\end{figure}
\end{frame}

\begin{frame}[c]
  \frametitle{Case II : Vertical Advective Velocity and Diffusion Coefficient}

\begin{figure}[ht!]
\centering
\includegraphics[width=0.8\textwidth]{AdvVelAndDiffCoeffEBSFail/I-129-VAdvVel.eps}
\caption{$^{129}I$.
For vertical advective velocities 
$6.31\times10^{-6}[m/yr]$ and above, lower reference diffusivities are 
ineffective at attenuating the mean of the peak doses for soluble, non-sorbing 
elements. 
}
\label{fig:VAdvVelI129VAdvVel}
\end{figure}
\end{frame}

\begin{frame}[c]
  \frametitle{Case II : Vertical Advective Velocity and Diffusion Coefficient}
\begin{figure}[htp!]
\centering
\includegraphics[width=0.8\textwidth]{AdvVelAndDiffCoeffEBSFail/Np-237.eps}
\caption{$^{237}Np$.
shows a very weak influence on peak annual dose 
rate for low reference diffusivities, but a direct proportionality between 
dose and reference diffusivity above a threshold.}
\label{fig:VAdvVelNp237}
\end{figure}
\end{frame}

\begin{frame}[c]
  \frametitle{Case II : Vertical Advective Velocity and Diffusion Coefficient}
\begin{figure}[ht!]
\centering
\includegraphics[width=0.8\textwidth]{AdvVelAndDiffCoeffEBSFail/Np-237-VAdvVel.eps}
\caption{$^{237}Np$.
shows a very weak influence on peak annual dose 
rate for low reference diffusivities, but a direct proportionality between 
dose and reference diffusivity above a threshold.}
\label{fig:VAdvVelNp237VAdvVel}
\end{figure}
\end{frame}

\begin{frame}[c]
  \frametitle{Case II : Vertical Advective Velocity and Diffusion Coefficient}
\begin{figure}[htp!]
\centering
\includegraphics[width=0.8\textwidth]{AdvVelAndDiffCoeffEBSFail/Se-79.eps}
\caption{$^{79}Se$.
$Se$ is non sorbing, but solubility limited.  
For low vertical advective velocity, 
the system is diffusion dominated.}
\label{fig:VAdvVelSe79}
\end{figure}
\end{frame}

\begin{frame}[c]
  \frametitle{Case II : Vertical Advective Velocity and Diffusion Coefficient}
\begin{figure}[ht!]
\centering
\includegraphics[width=0.8\textwidth]{AdvVelAndDiffCoeffEBSFail/Se-79-VAdvVel.eps}
\caption{$^{79}Se$.
$Se$ is non sorbing, but solubility limited.  
For high vertical advective 
velocity, the diffusivity remains important even in the advective regime as 
spreading facilitates transport in the presence of solubility limited transport. 
vertical advective velocity sensitivity.}
\label{fig:VAdvVelSe79VAdvVel}
\end{figure}
\end{frame}



\subsection{Solubility Coefficients}

This study varied the solubility coefficients for each isotope in the simulation 

The dissolution behavior of a solute in an aqueous solutions is called its 
solubility. This behavior is limited by the solute's solubility limit, described  
by an equilibrium constant that depends upon temperature, water chemistry, and 
the properties of the element. The equilibrium constant for many reactions are known, and can be found in 
chemical tables.

When $IAP/K<1$, the solution is undersaturated with respect to the products. When, 
conversely, $IAP/K>1$, the solution is oversaturated and precipitation of solids 
in the volume will occur. 

The solubility coefficients were varied in this simulation using a multiplier. 
This multiplier multiplied the most likely values of solubility for each 
element, so the relative solubility between elements was preserved.  Forty 
values of solubility coefficient multiplier were used to change the far field 
solubility.  The values of the solubility multiplier were deliberately varied 
over many magnitudes, from $1\time10^{-9}$ through $5\times10^{-10}$. 

For solubility limits below a certain threshold, the dose releases were directly 
proportional to the solubility limit, indicating that the radionuclide 
concentration saturated the groundwater up to the solubility limit near the 
waste form.  For solubility limits above the threshold, however, further 
increase to the limit had no effect on the peak dose. This demonstrates the 
situation in which the solubility limit is so high that even complete 
dissolution of the waste inventory into the pore water is insufficient to reach 
the solubility limit.

In Figures \ref{fig:SolSumFactor} and \ref{fig:SolSum}, it is clear that for 
solubility constants lower than a threshold, the relationship between peak 
annual dose and solubility limit is strong.



\begin{frame}[c]
  \frametitle{Case IV : Partition Coefficient}

This analysis investigated the peak dose rate contribution from various 
radionuclides to the partition coefficient of those radionuclides. 

The partition or distribution coefficient, $K_d$, relates the amount of contaminant adsorbed into the 
solid phase of the host medium to the amount of contaminant adsorbed into the 
aqueous phase of the host medium. It is a common empirical coefficient used to 
capture the effects of a number of retardation mechanisms. The coefficient 
$K_d$, in units of $[m^3\cdot kg^{-1}]$, is the ratio of the mass of contaminant in the 
solid to the mass of contaminant in the solution.
\end{frame}

\begin{frame}[c]
  \frametitle{Case IV : Partition Coefficient}
The retardation factor, $R_f$, which is the ratio between velocity of water through a 
volume and the velocity of a contaminant through that volume, can be expressed 
in terms of the partition coefficient,

\begin{align}
  R_f &= 1+\frac{\rho_b}{n_e}K_d
  \intertext{where}
  \rho_b &= ~~\mbox{bulk density}[kg\cdot m^{-3}]\nonumber
  \intertext{and}
  n_e &= ~~\mbox{effective porosity of the medium}[\%].\nonumber
\end{align}
\end{frame}

\begin{frame}[c]
  \frametitle{Case IV : Partition Coefficient}

The parameters in this model were all set to the default values except a multiplier 
applied to the partitioning $K_d$ coefficients.
\begin{table}[ht!]
\centering
\footnotesize{
\begin{tabular}{|l|l|l|r|r|}
\multicolumn{5}{c}{\textbf{Simulation Cases}}\\
\hline
\textbf{Case} & \textbf{Parameter} & \textbf{Units} & \textbf{Min. Value} & \textbf{Max. Value}\\
\hline
IV    & $K_{d,i}$    & $[m^3\cdot kg^{-1}]$       & $(1\times10^{-9})\langle K_{d,i}\rangle $    &  $(5\times10^{10})\langle K_{d,i}\rangle $ \\
\hline
\end{tabular}
\caption{Case IV varied the partitioning coefficient, $K_d$, multiplication 
  factor. This single parameter simulation case had 40 simulation 
groups of 100 realizations each.}
\label{tab:Cases}
}
\end{table}


\end{frame}

\begin{frame}[c]
  \frametitle{Case IV : Partition Coefficient}

\begin{figure}[ht]
\centering
\includegraphics[width=0.8\textwidth]{Sorption/Retardation_Summary_kdFactor.eps}
\caption{
For retardation coefficients greater than a threshold, the 
relationship between peak annual dose and retardation coefficient is a strong 
inverse one. }
\label{fig:KdSumFactor}
\end{figure}
\end{frame}

\begin{frame}[c]
  \frametitle{Case IV : Partition Coefficient}

\begin{figure}[ht]
\centering
\includegraphics[width=0.8\textwidth]{Sorption/Retardation_Summary_kd.eps}
\caption{
For retardation coefficients greater than a threshold, the 
relationship between peak annual dose and retardation coefficient is a strong 
inverse one. }
\label{fig:KdSum}
\end{figure}
\end{frame}



\begin{frame}[c]
  \frametitle{Case V : Waste Form Degradation Rate and Inventory}
These runs varied the waste form degradation rate and the waste inventory mass 
factor.  There were forty runs corresponding to eight values of the waste form degradation 
rate and five values of the mass factor.

\begin{table}[ht!]
\centering
\footnotesize{
\begin{tabular}{|l|l|l|r|r|}
\multicolumn{5}{c}{\textbf{Simulation Cases}}\\
\hline
\textbf{Case} & \textbf{Parameter} & \textbf{Units} & \textbf{Min. Value} & \textbf{Max. Value}\\
\hline
V     & $R_{WFDeg.}$           & $[yr^{-1}]$       & $10^{-9}$    &  $10^{-2}$ \\
      & Inventory              & [MTHM]         & $10^{-4}$    &  $10^1$ \\
\hline
\end{tabular}
\caption{Each dual and single parameter simulation case had 40 simulation 
groups of 100 realizations each.}
\label{tab:Cases}
}
\end{table}


\end{frame}

\begin{frame}[c]
  \frametitle{Case V : Waste Form Degradation Rate and Inventory}

Safety indicators for post closure repository performance have been developed by 
the \gls{UFD} campaign which utilize the inventory multiplier that was varied in 
this study \cite{nutt_generic_2009}. These indicators are normalized by a 
normalization factor (100 mrem/yr) recommended by the \gls{IAEA} as the limit to 
``relevant critical members of the public'' \cite{iaea_international_1996}. The functional form for 
this safety indicator for a single waste category, \gls{HLW}, is just 

\begin{align}
SI_{G} &= \left(\frac{\sum_{i=1}^{N}D_{G,i}(I_i, F_{d})}{100mrem/yr}\right)[GWe/yr].
\label{indicator}
\intertext{where}
SI_{G} &= \mbox{Safety indicator for disposal in media type G}[GWe/yr]\nonumber\\
N &= \mbox{Number of key radionuclides considered in this indicator}\nonumber\\
D_{G,i} &= \mbox{Peak dose rate from isotope i in media type G}[mrem/yr]\nonumber\\
F_{d} &= \mbox{Fractional waste form degradation rate}[1/yr].\nonumber
\end{align}
\end{frame}

\begin{frame}[c]
  \frametitle{Case V : Waste Form Degradation Rate and Inventory}
\begin{figure}[ht!]
\centering
\includegraphics[width=0.8\textwidth]{WFDegAndInv/I-129.eps}
\caption{
Highly soluble and non-sorbing $^{129}I$ demonstrates a direct proportionality between dose rate and 
fractional degradation rate until a turnover where other natural system 
parameters dampen transport.} 
\label{fig:WFDegI129}
\end{figure}
\end{frame}

\begin{frame}[c]
  \frametitle{Case V : Waste Form Degradation Rate and Inventory}

\begin{figure}[ht!]
\centering
\includegraphics[width=0.8\textwidth]{WFDegAndInv/I-129-MF.eps}
\caption{
Highly soluble and non-sorbing $^{129}I$ domonstrates a direct 
proportionality to the inventory multiplier.}
\label{fig:WFDegI129MF}
\end{figure}
\end{frame}

\begin{frame}[c]
  \frametitle{Case V : Waste Form Degradation Rate and Inventory}

\begin{figure}[ht!]
\centering
\includegraphics[width=0.8\textwidth]{WFDegAndInv/Cl-36.eps}
\caption{
Highly soluble and non-sorbing $^{36}Cl$ demonstrates a direct proportionality between dose rate and 
fractional degradation rate until a turnover where other natural system 
parameters dampen transport.} 
\label{fig:WFDegCl36}
\end{figure}

\end{frame}

\begin{frame}[c]
  \frametitle{Case V : Waste Form Degradation Rate and Inventory}
\begin{figure}[ht!]
\centering
\includegraphics[width=0.8\textwidth]{WFDegAndInv/Cl-36-MF.eps}
\caption{
Highly soluble and non-sorbing $^{129}I$ domonstrates a direct 
proportionality to the inventory multiplier.}
\label{fig:WFDegCl36MF}
\end{figure}
\end{frame}

\begin{frame}[c]
  \frametitle{Case V : Waste Form Degradation Rate and Inventory}
The peaks for solubility limited, sorbing elements such as $Tc$ and $Np$, on the 
other hand, have a more dramatic turnover.  For very high degradation rates, the 
dependence on mass factor starts to round off due to attenuation by solubility 
limits.


\begin{figure}[ht!]
\centering
\includegraphics[width=0.8\textwidth]{WFDegAndInv/Tc-99.eps}
\caption{
Solubility limited and sorbing $^{99}Tc$ demonstrates a direct proportionality 
to fractional degradation rate until attuation by its solubility limit and other 
natural system parameters. } 
\label{fig:WFDegTc99}
\end{figure}
\end{frame}

\begin{frame}[c]
  \frametitle{Case V : Waste Form Degradation Rate and Inventory}

\begin{figure}[ht!]
\centering
\includegraphics[width=0.8\textwidth]{WFDegAndInv/Tc-99-MF.eps}
\caption{
  Solubility limited and sorbing $^{99}Tc$ demonstrates a direct proportionality 
to fractional degradation rate until attuation by its solubility limit and other 
natural system parameters. } 
\label{fig:WFDegTc99MF}
\end{figure}

\end{frame}

\begin{frame}[c]
  \frametitle{Case V : Waste Form Degradation Rate and Inventory}

\begin{figure}[ht!]
\centering
\includegraphics[width=0.8\textwidth]{WFDegAndInv/Np-237.eps}
\caption{
  Solubility limited and sorbing $^{237}Np$ demonstrates a direct proportionality 
to fractional degradation rate until attuation by its solubility limit and other 
natural system parameters. } 
\label{fig:WFDegNp237}
\end{figure}
\end{frame}

\begin{frame}[c]
  \frametitle{Case V : Waste Form Degradation Rate and Inventory}

\begin{figure}[ht!]
\centering
\includegraphics[width=0.8\textwidth]{WFDegAndInv/Np-237-MF.eps}
\caption{
  Solubility limited and sorbing $^{237}Np$ demonstrates a direct proportionality 
to fractional degradation rate until attuation by its solubility limit and other 
natural system parameters. } 
\label{fig:WFDegNp237MF}
\end{figure}

\end{frame}


\begin{frame}[c]
  \frametitle{Case VI : Waste Package Failure Time and Diffusion Coefficient}

To investigate the effect of the waste package failure time, it was varied over 
five magnitudes from one thousand to ten million years. Simultaneously, the reference 
diffusivity was varied over the eight magnitudes between $1\times10^{-8}$ and 
$1\times10^{-15}$ in order to determine the correlation between increased 
radionuclide mobility and the waste package lifetime. 
\begin{table}[ht!]
\centering
\footnotesize{
\begin{tabular}{|l|l|l|r|r|}
\multicolumn{5}{c}{\textbf{Simulation Cases}}\\
\hline
\textbf{Case} & \textbf{Parameter} & \textbf{Units} & \textbf{Min. Value} & \textbf{Max. Value}\\
\hline 
VI    & $t_{WPFail}$        & $[yr]$         & $10^3$    &  $10^7$ \\
      & $D_{eff}$           & $[m^2\cdot s^{-1}]$       & $10^{-8}$    &  $10^{-5}$ \\
\hline
\end{tabular}
\caption{Case IV simultaneously varied the time of waste package failure and the 
  reference diffusivity. This dual parameter simulation case had 40 simulation 
groups of 100 realizations each.}
\label{tab:Cases}
}
\end{table}


\end{frame}

\begin{frame}[c]
  \frametitle{Case VI : Waste Package Failure Time and Diffusion Coefficient}

For the clay repository, the waste package failure time is entirely irrelevant 
until waste package failure times reach the million or ten million year time 
scale. 

\begin{figure}[ht!]
\centering
\includegraphics[width=0.8\textwidth]{WPFailExtended/I-129.eps}
\caption{$^{129}I$ waste package failure time sensitivity. }
\label{fig:WPFailI129}
\end{figure}
\end{frame}

\begin{frame}[c]
  \frametitle{Case VI : Waste Package Failure Time and Diffusion Coefficient}

\begin{figure}[ht!]
\centering
\includegraphics[width=0.8\textwidth]{WPFailExtended/I-129-WPFail.eps}
\caption{$^{129}I$ waste package failure time sensitivity. }
\label{fig:WPFailI129}
\end{figure}
\end{frame}





%%%%%%%%%%%%%%%%%%%%%%%%%%%%%%%%%%%%%%%%%%%%%%%%%%%%%%%%%%%%%%%%%%%%%%%%%%%%%%%%
This work is supported by the U.S. Department of Energy, Basic Energy Sciences, 
Office of Nuclear Energy, under con- tract # DE-AC02-06CH11357.


%%%%%%%%%%%%%%%%%%%%%%%%%%%%%%%%%%%%%%%%%%%%%%%%%%%%%%%%%%%%%%%%%%%%%%%%%%%%%%%%
\bibliographystyle{ans}
\bibliography{bibliography}
\end{document}
