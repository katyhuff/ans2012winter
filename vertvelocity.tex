\subsection{Case II : Vertical Advective Velocity and Diffusion Coefficient}

In this analysis, the threshold between primarily diffusive and primarily 
advective transport was investigated by varying the vertical advective velocity, 
$v_{adv,y}$, in conjunction with the diffusion coefficient, $D_{eff}$, as in 
Table \ref{tab:Cases}.
For the low diffusion coefficients and low advective 
velocities usually found in clay media, the model demonstrated behavior entirely 
in the diffusive regime. However, as the vertical advective velocity was 
increased, system behavior increasingly approached the advective regime. 

Soluble and non-sorbing nuclides $^{129}I$ and 
$^{36}Cl$ are more sensitive to vertical advective velocity for lower vertical 
advective velocities. It was shown that for vertical advective velocities 
above a threshold ($6.31\times10^{-6}[m/yr]$), lower reference diffusivities are 
ineffective at attenuating the mean of the peak doses for soluble, non-sorbing 
elements. 

Solubility limited and sorbing elements such as $Tc$ and $Np$ show a very weak 
influence on peak annual dose rate for low reference diffusivities, but show a 
direct proportionality between dose and reference diffusivity above a threshold. 
For $^{99}Tc$, for example, that threshold occurs at $1\times10^{-11}[m^2/s]$.  
Dose contribution from these elements show a proportional 
relationship with vertical advective velocity above a regime threshold at 
$6.31\times10^{-5}[m/yr]$, above which the system exhibits sensitivity to 
advection. 

%$Se$ is non sorbing, but solubility limited.  Results from $^{79}Se$ show that 
%for low vertical advective velocity, the system is diffusion dominated. However, 
%for high vertical advective velocity, the diffusivity remains important even 
%in the advective regime as spreading facilitates transport in the presence 
%of solubility limited transport. This indicates that for high vertical 
%advective velocity, the sorption properties of an element significantly 
%impact behavior.
