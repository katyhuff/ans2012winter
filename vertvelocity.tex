\subsection{Vertical Advective Velocity}

In this analysis, the threshold between primarily diffusive and primarily 
advective transport was investigated by varying the vertical advective velocity 
in conjunciton with the diffusion coefficient.  For the low 
diffusion coefficients and low advective velocities usually found in clay media, 
the model demonstrated behavior entirely in the diffusive regime, but as the 
vertical advective velocity grows, system behavior 
increasingly approached the advective regime. 

The diffusion coefficient was altered as in section \ref{sec:diffusivity} and 
the vertical advective velocity of the far field was sampled between $ 
6.31\times 10^{-8}$ and $ 6.31\times10^{-4} [m/yr]$. 

As in section \ref{sec:diffusivity}, in order to isolate the effect of the far 
field behavior, the waste form degradation rate was set to be very high as were 
the solubility and advective flow rate through the  \gls{EBS}. This guarunteed 
that in the first few time steps, the far field was the primary barrier to 
release. 

For isotopes of interest, higher advective velocity and higher diffusivity lead to higher 
means of the peak annual dose. However, the relationship between diffusivity and 
advective velocity adds depth to the notion of a boundary between diffusive and 
advective regimes.

The highly soluble and non-sorbing elements, $I$ and $Cl$ 
were expected to exhibit behavior that is highly sensitive 
to advection in the system in the advective regime but less sensitive to 
advection in the diffusive regime.  

In Figures \ref{fig:VAdvVelI129}, \ref{fig:VAdvVelI129VAdvVel}, 
\ref{fig:VAdvVelCl36}, and \ref{fig:VAdvVelCl36VAdvVel} , $^{129}I$ and 
$^{36}Cl$ are more sensitive to vertical advective velocity for lower vertical 
advective velocities. This demonstrates that for vertical advective velocities 
$6.31\times10^{-6}[m/yr]$ and above, lower reference diffusivities are 
ineffective at attenuating the mean of the peak doses for soluble, non-sorbing 
elements. 

The solubility limited and sorbing elements, $Tc$ and $Np$, in Figures 
\ref{fig:VAdvVelTc99}, \ref{fig:VAdvVelTc99VAdvVel}, \ref{fig:VAdvVelNp237}, and 
\ref{fig:VAdvVelNp237VAdvVel} show a very weak influence on peak annual dose 
rate for low reference diffusivities, but show a direct proportionality between 
dose and reference diffusivity above a threshold. For $^{99}Tc$, for example, 
that threshold occurs at $1\times10^{-11}[m^2/s]$. 

Dose contribution from $^{99}Tc$ has a proportional 
relationship with vertical advective velocity above a regime threshold at 
$6.31\times10^{-5}[m/yr]$, above which the system exhibits sensitivity to 
advection. 

The convergence of the effect of the reference diffusivity and vertical 
advective velocity for the cases above shows the effect of dissolved 
concentration (solubility) limits and sorption. $Se$ is non sorbing, but 
solubility limited.  The results from $^{79}Se$ in Figure \ref{fig:VAdvVelSe79} 
and \ref{fig:VAdvVelSe79VAdvVel} show that for low vertical advective velocity, 
the system is diffusion dominated.  However, for high vertical advective 
velocity, the diffusivity remains important even in the advective regime as 
spreading facilitates transport in the presence of solubility limited transport. 
