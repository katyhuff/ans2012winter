
\subsection{The Partition Coefficient}

This analysis investigated the peak dose rate contribution from various 
radionuclides to the partition coefficient of those radionuclides. 

The partition or distribution coefficient, $K_d$, relates the amount of contaminant adsorbed into the 
solid phase of the host medium to the amount of contaminant adsorbed into the 
aqueous phase of the host medium. It is a common empirical coefficient used to 
capture the effects of a number of retardation mechanisms. The coefficient 
$K_d$, in units of $[m^3\cdot kg^{-1}]$, is the ratio of the mass of contaminant in the 
solid to the mass of contaminant in the solution.

As indicated in Table \ref{tab:Cases}, the parameters in this model were all set 
to the default values except a multiplier applied to the partitioning $K_d$ 
coefficients. This multiplier preserved, in some sense, the widely varying 
relative sorption behavior among elements. Only the far field partition 
coefficients were altered by this factor. Partition coefficients effecting the 
EDZ and fast pathway were not changed.

The expected inverse relationship between the partition coefficient resulting 
peak annual dose was found for all elements that were not assumed to be 
effectively infinitely soluble.  It is clear from Figure \ref{fig:KdSum} that 
for partition coefficients greater than a threshold, the relationship between 
peak annual dose and partition coefficient is a strong inverse one. 

\begin{figure}[ht]
  \centering
  \includegraphics[width=\linewidth]{Partitioning_Summary.eps}
  \caption{$K_d$ sensitivity.  The peak annual dose due to an inventory, 
  $N$, of each isotope.}
  \label{fig:KdSum}
\end{figure}

